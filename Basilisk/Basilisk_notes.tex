% !TEX TS-program = pdflatex
% !TEX encoding = UTF-8 Unicode

% This is a simple template for a LaTeX document using the "article" class.
% See "book", "report", "letter" for other types of document.

\documentclass[12pt]{article} % use larger type; default would be 10pt

\usepackage[utf8]{inputenc} % set input encoding (not needed with XeLaTeX)

%%% Examples of Article customizations
% These packages are optional, depending whether you want the features they provide.
% See the LaTeX Companion or other references for full information.

%%% PAGE DIMENSIONS
\usepackage{geometry} % to change the page dimensions
\geometry{a4paper} % or letterpaper (US) or a5paper or....
% \geometry{margins=2in} % for example, change the margins to 2 inches all round
% \geometry{landscape} % set up the page for landscape
%   read geometry.pdf for detailed page layout information

\usepackage{graphicx} % support the \includegraphics command and options

% \usepackage[parfill]{parskip} % Activate to begin paragraphs with an empty line rather than an indent

%%% PACKAGES
\usepackage{booktabs} % for much better looking tables
\usepackage{array} % for better arrays (eg matrices) in maths
\usepackage{paralist} % very flexible & customisable lists (eg. enumerate/itemize, etc.)
\usepackage{verbatim} % adds environment for commenting out blocks of text & for better verbatim
\usepackage{subfig} % make it possible to include more than one captioned figure/table in a single float
% These packages are all incorporated in the memoir class to one degree or another...

%%% HEADERS & FOOTERS
\usepackage{fancyhdr} % This should be set AFTER setting up the page geometry
\pagestyle{fancy} % options: empty , plain , fancy
\renewcommand{\headrulewidth}{0pt} % customise the layout...
\lhead{}\chead{}\rhead{}
\lfoot{}\cfoot{\thepage}\rfoot{}

%%% SECTION TITLE APPEARANCE
\usepackage{sectsty}
\allsectionsfont{\sffamily\mdseries\upshape} % (See the fntguide.pdf for font help)
% (This matches ConTeXt defaults)

%%% ToC (table of contents) APPEARANCE
\usepackage[nottoc,notlof,notlot]{tocbibind} % Put the bibliography in the ToC
\usepackage[titles,subfigure]{tocloft} % Alter the style of the Table of Contents
\renewcommand{\cftsecfont}{\rmfamily\mdseries\upshape}
\renewcommand{\cftsecpagefont}{\rmfamily\mdseries\upshape} % No bold!

%%% END Article customizations
%%%package added by Wenge
\usepackage{soul}
\usepackage{graphicx}
\usepackage{listings}
\usepackage{xcolor}
\definecolor{mGreen}{rgb}{0,0.6,0}
\definecolor{mGray}{rgb}{0.5,0.5,0.5}
\definecolor{mPurple}{rgb}{0.58,0,0.82}
\definecolor{backgroundColour}{rgb}{0.95,0.95,0.92}

\lstdefinestyle{CStyle}{
    backgroundcolor=\color{backgroundColour},   
    commentstyle=\color{mGreen},
    keywordstyle=\color{magenta},
    numberstyle=\tiny\color{mGray},
    stringstyle=\color{mPurple},
    basicstyle=\footnotesize,
    breakatwhitespace=false,         
    breaklines=true,                 
    captionpos=b,                    
    keepspaces=true,                 
    numbers=left,                    
    numbersep=5pt,                  
    showspaces=false,                
    showstringspaces=false,
    showtabs=false,                  
    tabsize=2,
    language=C
}

%%% The "real" document content comes below...

\title{Notes Basilisk CFD}
\author{Wenge Huang}
%\date{} % Activate to display a given date or no date (if empty),
         % otherwise the current date is printed 

\begin{document}
\maketitle

\section{Codes}
\subsection{heights.h}
\hspace{5mm}
\textbf{
The “height-function” is a vector field which gives the distance, along each coordinate axis, from the center of the cell to the closest interface defined by a volume fraction field. }\par

That means if the vector is $(x, y ,z)$, \st{the results of the high function will be $(height(h.x[]), h.y, h.z)$}. In this case $h$ itself is a vector and we have the three component of $h$: $(h.x, h.y, h.z)$\par
\textbf{The distance is normalised with the cell size so that the coordinates of the interface are given by}\par
\textbf{
(x, y + Delta*height(h.y[])) or (x + Delta*height(h.x[]), y)
}Here heigh is a function that takes a scalar h.y[] or h.x[] as an input.
\par 
Here Delta is initially defined as the size of the cell
\vspace{5mm}

\begin{lstlisting}[style=CStyle]
#define HSHIFT 20.

static inline double height (double H) {
  return H > HSHIFT/2. ? H - HSHIFT : H < -HSHIFT/2. ? H + HSHIFT : H;
}

static inline int orientation (double H) {
  return fabs(H) > HSHIFT/2.;
}
\end{lstlisting}
\par
There are some other codes to calculate the distance and finally we can get the H which is distance. The we'll use the $height$ function in other place to call and get a good H (which is a scalar). 



\subsection{curvature.h}
To compute the curvature, we estimate the derivatives of the height functions in a given direction (x, y or z). We first check that all the heights are defined and that their orientations are the same. We then compute the curvature as:
$$
\kappa = \frac{h_{xx}}{(1 + h_x^2)^{3/2}}
$$
$$
\kappa = \frac{h_{xx}(1 + h_y^2) + h_{yy}(1 + h_x^2) - 2h_{xy}h_xh_y}
{(1 + h_x^2 + h_y^2)^{3/2}}
$$
\begin{lstlisting}[style=CStyle]
#include "heights.h"

#if dimension == 2
foreach_dimension()
static double kappa_y (Point point, vector h)
{
  int ori = orientation(h.y[]);
  for (int i = -1; i <= 1; i++)
    if (h.y[i] == nodata || orientation(h.y[i]) != ori)
      return nodata;
  double hx = (h.y[1] - h.y[-1])/2.;
  double hxx = (h.y[1] + h.y[-1] - 2.*h.y[])/Delta;
  return hxx/pow(1. + sq(hx), 3/2.);
}

foreach_dimension()
static coord normal_y (Point point, vector h)
{
  coord n = {nodata, nodata, nodata};
  if (h.y[] == nodata)
    return n;
  int ori = orientation(h.y[]);
  if (h.y[-1] != nodata && orientation(h.y[-1]) == ori) {
    if (h.y[1] != nodata && orientation(h.y[1]) == ori)
      n.x = (h.y[-1] - h.y[1])/2.;
    else
      n.x = h.y[-1] - h.y[];
  }
  else if (h.y[1] != nodata && orientation(h.y[1]) == ori)
    n.x = h.y[] - h.y[1];
  else
    return n;
  double nn = (ori ? -1. : 1.)*sqrt(1. + sq(n.x));
  n.x /= nn;
  n.y = 1./nn;
  return n;
}
\end{lstlisting}
\par
\textbf{We first check that all the heights are defined and that their orientations are the same.} we use the command foreach\_dimension().
with head file 'heights.h' we somehow calculate the height function h. That means we have h.x, h.y and h.z and also the corresponding orientations. \par
We now need to choose one of the $x$, $y$ or $z$ height functions to
compute the curvature. This is done by the function below which
returns the HF curvature given a volume fraction field *c* and a
height function field *h*.

\subsection{fractions.h}
\hspace{5mm}We need to use the fraction function to maintain or define volume and suface fractions either from an initial geometric definition or from an existing volume fraction field.\par
\begin{lstlisting}[style=CStyle]
fraction (f, sq(radius) - sq(y) - sq(x)); 
\end{lstlisting}\par
Now we're going to discuss how this function works.\par
The convenience \textbf{macros} below can be used to define volume and surface fraction fields directly from a function.\par
\begin{lstlisting}[style=CStyle]
#define fraction(f,func) do {			\
    vertex scalar phi[];			\
    foreach_vertex()				\
      phi[] = func;				\
    fractions (phi, f);				\
  } while(0)

#define solid(cs,fs,func) do {			\
    vertex scalar phi[];			\
    foreach_vertex()				\
      phi[] = func;				\
    fractions (phi, cs, fs);			\
  } while(0)
\end{lstlisting}\par
In this fraction function, it called the vertex variable phi[] and also another function fractions.
\begin{lstlisting}[style=CStyle]
struct Fractions {
  vertex scalar Phi; // compulsory
  scalar c;          // compulsory
  face vector s;     // optional
  double val;        // optional (default zero)
};

trace
void fractions (struct Fractions a)
{
  vertex scalar Phi = a.Phi;
  scalar c = a.c;
  face vector s = automatic (a.s);
  double val = a.val;
\end{lstlisting}\par

First, it defines a structure Fraction with members Phi (capital P) and c and others.
Then the structure is renamed as a. So we have a.Phi, a.c and so on. \par



\subsection{Example Height functions}
We need to use fraction function to get the volume fraction of the computation domain:
\begin{lstlisting}[style=CStyle]
scalar c[];
fraction (c, - (0.2 - sqrt(sq(x+0.2) + sq(y+0.2))));
\end{lstlisting}\par
With the volume fraction $c$ known, we can use the height function to calculate the height:
\begin{lstlisting}[style=CStyle]
 vector h[];
 heights (c, h);
\end{lstlisting}The heights() function takes a volume fraction field c and returns the height function vector field h.
\par
Also, we can use the height function to calculate the calvature:
\begin{lstlisting}[style=CStyle]
  scalar kappa[];
  curvature (c, kappa);
\end{lstlisting}







\end{document}