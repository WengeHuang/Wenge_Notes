% !TEX TS-program = pdflatex
% !TEX encoding = UTF-8 Unicode

% This is a simple template for a LaTeX document using the "article" class.
% See "book", "report", "letter" for other types of document.

\documentclass[11pt]{article} % use larger type; default would be 10pt

\usepackage[utf8]{inputenc} % set input encoding (not needed with XeLaTeX)

%%% Examples of Article customizations
% These packages are optional, depending whether you want the features they provide.
% See the LaTeX Companion or other references for full information.

%%% PAGE DIMENSIONS
\usepackage{geometry} % to change the page dimensions
\geometry{a4paper} % or letterpaper (US) or a5paper or....
% \geometry{margins=2in} % for example, change the margins to 2 inches all round
% \geometry{landscape} % set up the page for landscape
%   read geometry.pdf for detailed page layout information

\usepackage{graphicx} % support the \includegraphics command and options

% \usepackage[parfill]{parskip} % Activate to begin paragraphs with an empty line rather than an indent

%%% PACKAGES
\usepackage{booktabs} % for much better looking tables
\usepackage{array} % for better arrays (eg matrices) in maths
\usepackage{paralist} % very flexible & customisable lists (eg. enumerate/itemize, etc.)
\usepackage{verbatim} % adds environment for commenting out blocks of text & for better verbatim
\usepackage{subfig} % make it possible to include more than one captioned figure/table in a single float
% These packages are all incorporated in the memoir class to one degree or another...

%%% HEADERS & FOOTERS
\usepackage{fancyhdr} % This should be set AFTER setting up the page geometry
\pagestyle{fancy} % options: empty , plain , fancy
\renewcommand{\headrulewidth}{0pt} % customise the layout...
\lhead{}\chead{}\rhead{}
\lfoot{}\cfoot{\thepage}\rfoot{}

%%% SECTION TITLE APPEARANCE
\usepackage{sectsty}
\allsectionsfont{\sffamily\mdseries\upshape} % (See the fntguide.pdf for font help)
% (This matches ConTeXt defaults)

%%% ToC (table of contents) APPEARANCE
\usepackage[nottoc,notlof,notlot]{tocbibind} % Put the bibliography in the ToC
\usepackage[titles,subfigure]{tocloft} % Alter the style of the Table of Contents
\renewcommand{\cftsecfont}{\rmfamily\mdseries\upshape}
\renewcommand{\cftsecpagefont}{\rmfamily\mdseries\upshape} % No bold!

%%% END Article customizations
%%%package added by Wenge
\usepackage{esint}

%%% The "real" document content comes below...

\title{Notes for fluid dynamics}
\author{Wenge Huang}
%\date{} % Activate to display a given date or no date (if empty),
         % otherwise the current date is printed 

\begin{document}
\maketitle

\section{Basic equation}
\subsection{Substantial derivative}
\hspace{5mm}
$\frac{D}{Dt}$ is called the substantial derivative. $\frac{D\rho}{Dt}$ is the instantaneous time rate of change of density of the fluid element as it moves through a point. Here our eyes are locked on the fluid element as it is moving, and we are watching the density change of the element as it moves through the point. \par
This is different from $\frac{\partial \rho}{\partial t}$, which is physically the time rate of change of density at the fixed point. \par
$\frac{D\rho}{Dt}$ and $\frac{\partial \rho}{\partial t}$ are physically and numerically different quantities. \par
Here we show the definition of $\frac{D}{Dt}$. If we have
$$
\rho_{1} = \rho_{1}(x_{1}, y_{1}, z_{1}, t_{1})
$$
$$
\rho_{2} = \rho_{2}(x_{2}, y_{2}, z_{2}, t_{2})
$$
Then we calculate that
\begin{equation}
\frac{D\rho}{Dt} = \lim_{t_{1} \to t_{2}}\frac{\rho_{2}-\rho_{1}}{t_{1}-t_{2}}
\end{equation}
\begin{equation}
\frac{D\rho}{Dt} = \frac{\partial \rho}{\partial t}+u\frac{\partial \rho}{\partial x}+v\frac{\partial \rho}{\partial y}+w\frac{\partial \rho}{\partial z}
\end{equation}
\begin{equation}
\frac{D}{Dt} = \frac{\partial }{\partial t}+\vec{v}\cdot \nabla
\end{equation}
The first part is the local derivative and the second part is the convective derivative.\par
The substantial derivative is essentially the total derivative from calculus. For example:
$$
d\rho = \frac{\partial \rho}{\partial t}dt+\frac{\partial \rho}{\partial x}dx+\frac{\partial \rho}{\partial y}dy+\frac{\partial \rho}{\partial z}dz
$$
$$
\frac{D\rho}{Dt} = \frac{d \rho}{d t} = \frac{\partial \rho}{\partial t}+\frac{\partial \rho}{\partial x}\frac{dx}{dt}+\frac{\partial \rho}{\partial y}\frac{dy}{dt}+\frac{\partial \rho}{\partial z}\frac{dz}{dt}
$$



\subsection{Conservation of mass}
\hspace{5mm} We consider the finite control volume (here we consider that the CV is fixed in space, which means the shape of the volume is unchanged).\par
Mass is conserved: \textbf{Net mass flow out of the control volume through surface $S$ is equal to the time rate of decrease of mass inside the control volume $V$}.\par
The mass flux across a surface element $d\vec{S}$ is:
$$
\rho\vec{v}\cdot d\vec{S}
$$
The net mass flow out of control volume is:
$$
\iint_{S}\rho\vec{v}\cdot d\vec{S}
$$
The total mass inside the control volume is:
$$
\iiint_{V} \rho dV
$$
Since we consider the fixed control volume, the time rate of increase of mass inside the CV $V$ is then:
$$
\frac{\partial}{\partial t}\iiint_{V} \rho dV
$$Note that the time rate of decrease is negative, while the net mass flux out is positive. When considering conservation, we should make the sign correct.\par
Thus we have the conservation of mass:
\begin{equation}
\frac{\partial}{\partial t}\iiint_{V} \rho dV + \iint_{S}\rho\vec{v}\cdot d\vec{S} = 0
\end{equation}\par
Now we consider the control volume moving with the fluid. The control volume is always made up of the same fluid particles as it moves with the flow. The mass is fixed, but the volume $V$ and the control surface $S$ is always changing.\par
The total mass of the finite control volume is:
$$
M = \iiint_{\Omega} \rho d\Omega
$$
The volume integral is taken over the whole moving control volume $\Omega$. But the control volume is changing as the control volume moves downstream.\par
Since the mass $M$ is constant, it doesn't change with time, mathematically:
$$
\frac{dM}{dt} = 0
$$\par
Here we use the form of the substantial derivative:
\begin{equation}
\frac{D}{Dt} \iiint_{\Omega} \rho d\Omega =0
\end{equation}
Again, we get the integral form of the continuity equation (nonconservation form). We have obtained the \textbf{integral} form of the continuity equation in two different ways. Now we are trying to get the \textbf{differential} form also in two different ways. Note that it is not a simple transfer from the integral to differential mathematically. \par
This time, we don't consider the finite control volume $V$ but an infinitesimally small element $dxdydz$ (fixed in space).\par
The Net mass flux for the infinitesimally small element in $x$ is:
$$
\left[ \rho u + \frac{\partial \rho u}{\partial x}dx\right] \cdot dydz 
-(\rho u) dydz = \frac{\partial \rho u}{\partial x}dxdydz
$$
$$
\left[ \rho v + \frac{\partial \rho v}{\partial y}dx\right] \cdot dxdz 
-(\rho v) dxdz = \frac{\partial \rho v}{\partial y}dxdydz
$$
$$
\left[ \rho w + \frac{\partial \rho w}{\partial z}dz\right] \cdot dydx 
-(\rho w) dydx = \frac{\partial \rho w}{\partial z}dxdydz
$$As a result, the total net mass flow is:
$$
\left[\frac{\partial (\rho u)}{\partial x}+\frac{\partial (\rho v)}{\partial y}+ \frac{\partial (\rho w)}{\partial z}\right]dxdydz
$$\par
The time rate of mass is:
$$
\frac{\partial \rho}{\partial t}dxdydz
$$\par
Finally, we get the differential form of the continuity equation:
\begin{equation}
\frac{\partial \rho}{\partial t} +\frac{\partial (\rho u)}{\partial x}+\frac{\partial (\rho v)}{\partial y}+ \frac{\partial (\rho w)}{\partial z} = 0
\end{equation}
$$
\frac{\partial \rho}{\partial t} +\nabla \cdot(\rho \vec{v})=0
$$\par
The moving control volume mode. The mass of the infinitesimally small fluid element is:
$$
\delta m = \rho \delta \Omega
$$
Since the mass is conserved, we have:
$$
\frac{D (\delta m)}{Dt} = \frac{D (\rho \delta \Omega)}{Dt}= 0
$$
The derivative by part allows us to get:
$$
\frac{D \rho}{Dt} + \frac{\rho}{\delta \Omega}\cdot\frac{D ( \delta \Omega)}{Dt}= 0
$$\par
The volume can be calculated as:
$$
\delta \Omega = \iint_{\delta S}\vec{v}dS \cdot \delta t
$$
$$
\frac{D ( \delta \Omega)}{Dt} =\lim_{\delta t \to 0} \frac{\delta \Omega}{\delta t}=\lim_{\delta s \to 0}\iint_{\delta S}\vec{v}dS =\lim_{\delta \Omega \to 0}\iiint_{\delta \Omega} (\nabla \cdot \vec{v})d \Omega = (\nabla \cdot \vec{v})\delta \Omega
$$\par
As a result, we find another form of the continuity equation:
\begin{equation}
\frac{D \rho}{Dt} + \rho  (\nabla \cdot \vec{v})= 0
\end{equation}
Note that
$$
\frac{D \rho}{Dt} + \rho  (\nabla \cdot \vec{v}) = \frac{\partial \rho}{\partial t} + \vec{v} \cdot\nabla \rho + \rho  (\nabla \cdot \vec{v}) = \frac{\partial \rho}{\partial t} + \nabla \cdot(\rho \vec{v})=0
$$



\subsection{Conservation of momentum}
\hspace{5mm} We discuss the surface force first. Unlike body force which can be moved to a mass center. The surface force depends on the surface area, more specifically: the surface area and normal vector. We can say that what specific unit surface area is the normal vector $\vec{n}$. Force is another vector. Thus tensor is an operator \textbf{which maps from one vector (the normal vector of the surface) to another vector (the force vector)}. \par
Consider a fluid volume with characteristic length $R$. The body force is proportional to the volume which scales as $R^{3}$. The surface force is proportional to the total surface area, which scales as $R^{2}$. Thus, as $R$ approaches zero, the total body force can not be balanced by the total surface force. Thus, we should consider the body force and the surface force individually/independently.\par
 The stress tensor is symmetrical:
 $$
 \sigma_{kj} =  \sigma_{jk} 
 $$\par
 The stress tensor $ \sigma_{ij}$ can be expressed as the sum of an isotropic part and a non-isotropic part:
$$
  \sigma_{ij} = -p\delta_{ij} +d_{ij}
 $$
 For a Newtonian fluid, it can be expressed as:
 $$
 \sigma = -pI +T
 $$
$$
\left(
\begin{array} {ccc}
\sigma_{xx} & \tau_{xy} & \tau_{xz} \\
\tau_{yx} & \sigma_{yy} & \tau_{yz} \\
\tau_{zx} & \tau_{zy} &\sigma_{zz}
\end{array}
\right)
=-
\left(
\begin{array} {ccc}
p & 0 & 0 \\
0 & p & 0 \\
0 & 0 &p
\end{array}
\right) +
\left(
\begin{array} {ccc}
\sigma_{xx} + p & \tau_{xy} & \tau_{xz} \\
\tau_{yx} & \sigma_{yy} + p& \tau_{yz} \\
\tau_{zx} & \tau_{zy} &\sigma_{zz}+ p
\end{array}
\right)
$$
$$
\left(
\begin{array} {ccc}
\sigma_{xx} & \tau_{xy} & \tau_{xz} \\
\tau_{yx} & \sigma_{yy} & \tau_{yz} \\
\tau_{zx} & \tau_{zy} &\sigma_{zz}
\end{array}
\right)
=-
\left(
\begin{array} {ccc}
p & 0 & 0 \\
0 & p & 0 \\
0 & 0 &p
\end{array}
\right) +
\left(
\begin{array} {ccc}
\tau_{xx}  & \tau_{xy} & \tau_{xz} \\
\tau_{yx} & \tau_{yy}& \tau_{yz} \\
\tau_{zx} & \tau_{zy} &\tau_{zz}
\end{array}
\right)
$$
\par
The $i-$compoment of the surface force on a small surface area $\delta S$ is $\sigma_{ij}n_{j}\delta S$. We integrate over the surface to obtain:
$$
\iint_{S}\sigma_{ij}n_{j}\delta S = \iiint_{V} \frac{\partial \sigma_{ij}}{\partial x_{j}}dV
$$\par 
The change of momentum is expressed as:
$$
\frac{D}{Dt}\iiint_{V} \rho \vec{v}dV 
$$\par
Since $\frac{D}{Dt}$ is applied to the fluid element. When it is applied to the integral, we can apply the substantial derivative to each fluid element and then integrate them together. 
$$
\frac{D}{Dt}\iiint_{V} \rho \vec{v}dV  = \\frac{D(\rho \vec{v}dV)}{Dt}  = \iiint_{V}\rho \frac{D\vec{v}}{Dt} dV+\iiint_{V}\vec{v}\frac{D(\rho dV)}{Dt} = \iiint_{V}\rho \frac{D\vec{v}}{Dt} dV
$$Since $\frac{D(\rho dV)}{Dt}$ = 0.\par
This can be applied to arbitrary quantities:
$$
\frac{D}{Dt}\iiint_{V} \rho \theta dV = \iiint_{V}\rho \frac{D\theta}{Dt} dV
$$\par
Now we obtain the equation for the conservation of momentum:
\begin{equation}
\iiint_{V}\rho \frac{D\vec{v}_{i}}{Dt} dV = \iiint_{V}F_{i}\rho dV + \iiint_{V} \frac{\partial \sigma_{ij}}{\partial x_{j}}dV
\end{equation}
$$
\left(
\begin{array} {ccc}
\sigma_{xx} & \tau_{xy} & \tau_{xz} \\
\tau_{yx} & \sigma_{yy} & \tau_{yz} \\
\tau_{zx} & \tau_{zy} &\sigma_{zz}
\end{array}
\right)
=-
\left(
\begin{array} {ccc}
p & 0 & 0 \\
0 & p & 0 \\
0 & 0 &p
\end{array}
\right) +
\left(
\begin{array} {ccc}
\tau_{xx}  & \tau_{xy} & \tau_{xz} \\
\tau_{yx} & \tau_{yy}& \tau_{yz} \\
\tau_{zx} & \tau_{zy} &\tau_{zz}
\end{array}
\right)
$$
We have the differential form:
$$
\rho \frac{Du}{Dt} = - \frac{\partial p}{\partial x}+\frac{\partial \tau_{xx}}{\partial x}+\frac{\partial \tau_{xy}}{\partial y} + \frac{\partial \tau_{xz}}{\partial z} +\rho F_{x}
$$
$$
\rho \frac{Dv}{Dt} = - \frac{\partial p}{\partial y}+\frac{\partial \tau_{yx}}{\partial x}+\frac{\partial \tau_{yy}}{\partial y} + \frac{\partial \tau_{yz}}{\partial z} +\rho F_{y}
$$
$$
\rho \frac{Dw}{Dt} = - \frac{\partial p}{\partial z}+\frac{\partial \tau_{zx}}{\partial x}+\frac{\partial \tau_{zy}}{\partial y} + \frac{\partial \tau_{zz}}{\partial z} +\rho F_{z}
$$\par
The left-hand side substantial derivative can be written as:
$$
\rho \frac{Du}{Dt} = \rho \left(  \frac{\partial u}{\partial t}+\vec{v}\cdot \nabla u\right) = \frac{\partial (\rho u)}{\partial t}-u \frac{\partial \rho}{\partial t} +\vec{v}\cdot \rho\nabla u=\frac{\partial (\rho u)}{\partial t}-u \frac{\partial \rho}{\partial t} +\nabla (\rho u\vec{v}) - u\nabla (\rho\vec{v})
$$
$$
\frac{\partial (\rho u)}{\partial t}-u \frac{\partial \rho}{\partial t} +\nabla (\rho u\vec{v}) - u\nabla (\rho\vec{v})=\frac{\partial (\rho u)}{\partial t}+\nabla (\rho u\vec{v}) -u[\frac{\partial \rho}{\partial t} +\nabla (\rho\vec{v})] = \frac{\partial (\rho u)}{\partial t}+\nabla (\rho u\vec{v})
$$Thus, we get the conservation form of the Navier-Stokes equation:
$$
\frac{\partial (\rho u)}{\partial t}+\nabla (\rho u\vec{v}) = - \frac{\partial p}{\partial x}+\frac{\partial \tau_{xx}}{\partial x}+\frac{\partial \tau_{xy}}{\partial y} + \frac{\partial \tau_{xz}}{\partial z} +\rho F_{x}
$$
$$
\frac{\partial (\rho v)}{\partial t}+\nabla (\rho v\vec{v}) = - \frac{\partial p}{\partial y}+\frac{\partial \tau_{yx}}{\partial y}+\frac{\partial \tau_{yy}}{\partial y} + \frac{\partial \tau_{yz}}{\partial z} +\rho F_{y}
$$
$$
\frac{\partial (\rho w)}{\partial t}+\nabla (\rho w\vec{v}) = - \frac{\partial p}{\partial z}+\frac{\partial \tau_{zx}}{\partial x}+\frac{\partial \tau_{zy}}{\partial y} + \frac{\partial \tau_{zz}}{\partial z} +\rho F_{z}
$$\par
When the fluid is Newtonian, we have:
$$
\tau_{xx} = \lambda(\nabla \cdot \vec{v})+2\mu \frac{\partial u}{\partial x}
$$
$$
\tau_{yy} = \lambda(\nabla \cdot \vec{v})+2\mu \frac{\partial v}{\partial y}
$$
$$
\tau_{zz} = \lambda(\nabla \cdot \vec{v})+2\mu \frac{\partial w}{\partial z}
$$
$$
\tau_{xy}=\tau_{yx}=\mu \left[\frac{\partial v}{\partial x}+\frac{\partial u}{\partial y} \right]
$$
$$
\tau_{xz}=\tau_{zx}=\mu \left[\frac{\partial w}{\partial x}+\frac{\partial u}{\partial z} \right]
$$
$$
\tau_{yz}=\tau_{zy}=\mu \left[\frac{\partial v}{\partial z}+\frac{\partial w}{\partial y} \right]
$$where $\mu$ is the molecular viscosity coefficient and $\lambda$ is the second viscosity coefficient. Stokes made the hypothesis that:
$$
\lambda = -\frac{2}{3} \mu
$$\par
When the fluid is incompressible, which means the density is constant $\frac{D \rho}{Dt} = 0$, we have the gradient of the velocity is zero $\nabla \cdot \vec{v} = 0$. With some assumptions we have:
$$
\frac{\partial }{\partial x}\mu \left[\frac{\partial u}{\partial x}+\frac{\partial u}{\partial x} \right]+\frac{\partial }{\partial y}\mu \left[\frac{\partial v}{\partial x}+\frac{\partial u}{\partial y} \right] + \frac{\partial }{\partial z}\mu \left[\frac{\partial w}{\partial x}+\frac{\partial u}{\partial z} \right]
$$
$$
\mu \left[\frac{\partial^{2} u}{\partial x^{2}} + \frac{\partial^{2} u}{\partial y^{2}}+\frac{\partial^{2} u}{\partial z^{2}}  \right]+\mu\left[\frac{\partial }{\partial x}\frac{\partial u}{\partial x}+\frac{\partial }{\partial y}\frac{\partial v}{\partial x} +\frac{\partial }{\partial z}\frac{\partial w}{\partial x}\right]
$$
$$
\mu \left[\frac{\partial^{2} u}{\partial x^{2}} + \frac{\partial^{2} u}{\partial y^{2}}+\frac{\partial^{2} u}{\partial z^{2}}  \right]+\mu\frac{\partial }{\partial x}\left[\nabla \cdot \vec{v}  \right]=\mu \nabla^{2}u
$$\par 
Thus, for the incompressible flow, we have:
\begin{equation}
\rho \frac{D \vec{v}}{Dt} = -\nabla p + \mu \nabla^{2}\vec{v} +\rho \vec{F}
\end{equation}
\begin{equation}
\rho \frac{\partial \vec{v}}{\partial t} + {\rho (\vec{v} \cdot \nabla)\vec{v}} = -\nabla p + \mu \nabla^{2}\vec{v} +\rho \vec{F}
\end{equation}



\subsection{Conservation of energy}






































\end{document}



























