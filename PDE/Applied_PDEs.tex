% !TEX TS-program = pdflatex
% !TEX encoding = UTF-8 Unicode

% This is a simple template for a LaTeX document using the "article" class.
% See "book", "report", "letter" for other types of document.

\documentclass[12pt]{article} % use larger type; default would be 10pt

\usepackage[utf8]{inputenc} % set input encoding (not needed with XeLaTeX)

%%% Examples of Article customizations
% These packages are optional, depending whether you want the features they provide.
% See the LaTeX Companion or other references for full information.

%%% PAGE DIMENSIONS
\usepackage{geometry} % to change the page dimensions
\geometry{a4paper} % or letterpaper (US) or a5paper or....
% \geometry{margins=2in} % for example, change the margins to 2 inches all round
% \geometry{landscape} % set up the page for landscape
%   read geometry.pdf for detailed page layout information

\usepackage{graphicx} % support the \includegraphics command and options

% \usepackage[parfill]{parskip} % Activate to begin paragraphs with an empty line rather than an indent

%%% PACKAGES
\usepackage{booktabs} % for much better looking tables
\usepackage{array} % for better arrays (eg matrices) in maths
\usepackage{paralist} % very flexible & customisable lists (eg. enumerate/itemize, etc.)
\usepackage{verbatim} % adds environment for commenting out blocks of text & for better verbatim
\usepackage{subfig} % make it possible to include more than one captioned figure/table in a single float
% These packages are all incorporated in the memoir class to one degree or another...

%%% HEADERS & FOOTERS
\usepackage{fancyhdr} % This should be set AFTER setting up the page geometry
\pagestyle{fancy} % options: empty , plain , fancy
\renewcommand{\headrulewidth}{0pt} % customise the layout...
\lhead{}\chead{}\rhead{}
\lfoot{}\cfoot{\thepage}\rfoot{}

%%% SECTION TITLE APPEARANCE
\usepackage{sectsty}
\allsectionsfont{\sffamily\mdseries\upshape} % (See the fntguide.pdf for font help)
% (This matches ConTeXt defaults)

%%% ToC (table of contents) APPEARANCE
\usepackage[nottoc,notlof,notlot]{tocbibind} % Put the bibliography in the ToC
\usepackage[titles,subfigure]{tocloft} % Alter the style of the Table of Contents
\renewcommand{\cftsecfont}{\rmfamily\mdseries\upshape}
\renewcommand{\cftsecpagefont}{\rmfamily\mdseries\upshape} % No bold!

%%% END Article customizations
%%% package added by Wenge
\usepackage{derivative}
\usepackage{mathtools}

\usepackage{amsfonts}
%%% The "real" document content comes below...

\title{Notes for Applied PDEs}
\author{Wenge Huang}
%\date{} % Activate to display a given date or no date (if empty),
         % otherwise the current date is printed 

\begin{document}
\maketitle

\section{Lecture1} 


\hspace{5mm}
Let's consider the 2nd order PDEs. The most general form of a 2nd order PDE with two variables is:\\
\begin{equation}  
Au_{xx}+Bu_{xy}+Cu_{yy}+Du_{x}+Eu_{y}+Fu=G
 % there should be no gap between any of the two rows  
\end{equation} \\
where  $A, B, ... G$ are constants or given functions of $x$ and $y$.\par
\subsection{Types of PDEs}
\hspace{5mm}All linear PDEs in the form of Eq(1) can be classified into three types:  \emph {\textbf{hyperbolic, parabolic}} and \emph{\textbf{elliptic}}.\\

The hyperbolic type: e.g. the wave equation 
\begin{equation}
\Delta (x, y) = B^{2}(x, y) - 4A(x, y)C(x, y) >0
\end{equation}
\par
The parabolic type: e.g. the heat equation
\begin{equation}
\Delta (x, y) = B^{2}(x, y) - 4A(x, y)C(x, y) =0
\end{equation}
\par
The elliptic type: e.g. the Laplace equation
\begin{equation}
\Delta (x, y) = B^{2}(x, y) - 4A(x, y)C(x, y) <0
\end{equation}\par


\subsection{Superposition principle}
\hspace{5mm}
If $u_{1}$, $u_{2}$, ... $u_{n}$, are the solutions to the linear homogeneous PDE $Lu = 0$, and $u_{1}$, $u_{2}$, ... $u_{n}$ $\in$ $R$. Here $L$ is a linear differential operator.  Then the linear combination of $\sum_{1}^{n}c_{i}u_{i}$ is also the solution of the PDE.\par
Let $S_{h}$ be the set of all solutions to the homogeneous problem $$Lu = 0$$ \par
Then we consider the inhomogeneous problem$$Lu = f$$\par
The set of all solutions to this inhomogeneous problem is given by
$$
S_{i}=\{u_{i}+u_{h}| u_{h} \in S_{h}\}
$$Here $u_{i}$ is a particular solution to the inhomogeneous problem and $S_{i}$ is the translation of $S_{h}$ by $u_{i}$.\par
\section{Lecture2}


\subsection{ODE}
\hspace{5mm}Let's consider the 2nd ODE first. 
\begin{equation}
ax''(t) + bx'(t) +cx(t) =0
\end{equation}where $a$, $b$ and $c$ $\in$ $R$ and $a \neq 0$.\par
We consider the characteristic equation first:
\begin{equation}
a\lambda^{2} + b\lambda +c =0
\end{equation}with two solution $\lambda_{1}$ and $\lambda_{2}$.\par
Case I, when $\lambda_{1} \neq \lambda_{2}$, we have the independent solutions:
$$\left \{ \begin{array}{rcl}
x_{1}(t) = e^{\lambda_{1}t} \\
x_{2}(t) = e^{\lambda_{2}t}
\end{array}\right.
$$.\par
Case II, when have the same roots $\lambda_{1} = \lambda_{2}$, we have the solution:
$$\left \{ \begin{array}{lcc}
x_{1}(t) = e^{\lambda_{1}t} \\
x_{2}(t) =\emph{\textbf{t}}e^{\lambda_{2}t}
\end{array}\right.
$$.\par
Case III, we have complex conjugate pairs of roots $\lambda_{1} = \alpha +\beta i$ and $\lambda_{2} = \alpha -\beta i$. The two independent  solutions are:
$$\left \{ \begin{array}{lcc}
x_{1}(t) = e^{\alpha t}cos(\beta t) \\
x_{2}(t) =e^{\alpha t}sin(\beta t)
\end{array}\right.
$$.\par
With initial conditions given, we'll search solution with a linear combination of the independent solutions
\begin{equation}
x(t) = C_{1}x_{1}(t) + C_{2}x_{2}(t)
\end{equation} where $C_{1}$ and $C_{2}$ $\in$ $R$. \par
This is called the general solution of the homogeneous problem Eq(5).\par
Then for the inhomogeneous version, 
\begin{equation}
ax''(t) + bx'(t) +cx(t) =f(t)
\end{equation}
we need to use the \textbf {variable of parameter formula} to find a particular solution. \par



\section{Heat equation}
\hspace{5mm}Let's consider the 1D heat equation on the interval $(0, L)$ and subject to some initial conditions(IBVP).\par
%% Partial derivative symbol in LaTeX \frac{\partial f}{\partial x_i}
\begin{equation}
\left \{ \begin{array}{llc}
\pdv{u}{t} =\frac{\partial^{2}u}{\partial x^{2}} & x \in (0,L),  t>0 \\
u(0,t) = 0, u(L, t) = 0& t>0 \\
u(x, 0) = f(x) & 0 \leq x \leq L
\end{array}\right.
\end{equation}


\subsection{Separation of variable}
\hspace{5mm}We are looking for non-trivial solutions. We assume:
$$
u(x,t) = V(x)T(t) \hspace{10mm} x \in (0,L),  t>0
$$\par
Plugging it into the Eq(9), we can obtain
$$
\frac{T'}{T} = \frac{V''}{V} = \beta \hspace{10mm} \forall x \in (0,L),  t>0
$$ Thus, we get
$$
\left\{\begin{array}{lcc}
T'(t) = \beta T(t)\\
V''(t) = \beta V(t)
\end{array}\right.
$$ we successfully transfer the PDE to ODEs. \par
Considering the boundary conditions $u(0,t) = 0, u(L, t) = 0, \hspace{5mm} t>0 $ we have $V(0) = 0$ and $V(L) = 0$.\par
$$
\left\{\begin{array}{lcc}
V''(t) = \beta V(t) & ,x \in (0,L)\\
V(0) = 0 \\
V(L) = 0
\end{array}\right.
$$ In this case, the characteristic equation is:
$$
\lambda^{2} - \beta = 0
$$we have three cases for $\beta$. And we can check that only when $\beta <0$, the solution is non-trivial. \par 
When $\beta >0$ we have two distinguished real roots
$$\left \{ \begin{array}{lcc}
V_{1}(x) = e^{-\sqrt{\beta}x} \\
V_{2}(x) = e^{\sqrt{\beta}x}
\end{array}\right.
$$\par
Then $V(x) = C_{1}V_{1}(x)+C_{2}V_{2}(x)$ with the boundary conditions $V(0) = 0$ and $V(L) = 0$. As a result $C_{1} = C_{2} =0$.\par
When $\beta =0$, we have $\lambda_{1} = \lambda_{2} = 0$. Thus we can find that $V_{1}(x)=1$ and $V_{1}(x)=x$. Considering the boundary conditions, the coefficients are also should be 0, \emph{i.e.} $C_{1} = C_{2} =0$. 
When $\beta <0$, the solutions of the characteristic function is 
$$
\lambda = \pm \sqrt{-\beta}i
$$As a result, we find that $V_{1}(x) = cos (\sqrt{-\beta }x)$ and $V_{2}(x) = sin (\sqrt{-\beta }x)$.
With the boundary condition given $V(0) = 0$ and $V(L) = 0$, we have that:
$$
C_{1} + 0 = 0
$$
$$
C_{1} cos (\sqrt{-\beta }L) + C_{2} sin (\sqrt{-\beta }L) = 0
$$
For non-trivial solutions, we need to have $sin (\sqrt{-\beta }L) = 0$ and 
$$
\beta_{n} = - \left(
\frac{n \pi}{L}
\right)^{2}, \hspace{5mm} n \in N
$$Thus we find the eigenvalues and the eigenfunctions:
\begin{equation}
 \left\{
 \begin{array}{lcc}
 \beta_{n} = - \left(
\frac{n \pi}{L}
\right)^{2} &, n \in N \\
V_{n}(x) = sin(\frac{n\pi}{L}x)
 \end{array} \right.
\end{equation}
Now for each $\beta$ we have the solution for $T(t)$:
$$
T'(t) = \beta_{n}T(t), \hspace{5mm}  n \in N
$$
$$
T(t) = e^{ - \left(
\frac{n \pi}{L}
\right)^{2}t} \hspace{5mm} n \in N
$$
We find the solution:
\begin{equation}
u_{n}(x, t) = V_{n}(x)T_{n}(t) = e^{ - \left(
\frac{n \pi}{L}
\right)^{2}t}sin(\frac{n\pi}{L}x) \hspace{5mm} n\in N
\end{equation}\par
From the above calculation, we know that each $u_{n}$ satisfies the following homogeneous BVP:
\begin{equation}
\left \{ \begin{array}{llc}
\pdv{u}{t} =\frac{\partial^{2}u}{\partial x^{2}} & & x \in (0,L),  t>0 \\
u(0,t) = 0, u(L, t) = 0 && t>0 
\end{array}\right.
\end{equation}\par
We look for the solution of the form:
\begin{equation}
u(x, t) = \sum_{n=1}^{\infty}C_{n}u_{n}(x, t)
\end{equation}\par
We need to utilize the initial conditions to find the coefficient. The initial condition is:
$$
u(x, 0) = f(x) \hspace{5mm} x \in [0, L]
$$
With $t = 0$, we have $T(0) = 1$ and also:
\begin{equation}
\sum_{n=1}^{\infty}C_{n}sin(\frac{n\pi}{L}x) = f(x)
\end{equation}\par
This is exactly the problem of finding the \textbf{Fourier sine expansion} of the given function $f$.\par
To find the coefficient $C_{n}$, we use the fact that $V_{n}$ are orthogonal to each other in the sense that:
\begin{equation}
\int_{0}^{L}V_{n}(x)V_{m}(x)dx = 
\left\{
\begin{array}{lcc}
0 & if & m \neq n \\
\frac{L}{2} & if& m = n
\end{array}
\right.
\end{equation}\par
Now multiply both side of Eq(14) by $V_{m}$ and integrate from 0 to $L$.
$$
\int_{0}^{L}\sum_{n=1}^{\infty}C_{n}V_{n}(x)V_{m}(x)dx = \int_{0}^{L}f(x)V_{m}(x)dx
$$
Assume we can switch $\int_{0}^{L}$ with $\sum_{n=1}^{\infty}$, we get
$$
C_{n} = \frac{2}{L}\int_{0}^{L}f(x)sin(\frac{n\pi}{L}x)dx \hspace{5mm} n\in N
$$\par
With known the coefficient, we finally obtain the complete solution of the IBVP Eq(9):
\begin{equation}
u(x, t) = \sum_{n=1}^{\infty}\frac{2}{L}\int_{0}^{L}f(x')sin(\frac{n\pi}{L}x')dx' e^{ - \left(
\frac{n \pi}{L}
\right)^{2}t}sin(\frac{n\pi}{L}x) \hspace{5mm} n \in N, t>0, x \in [0, L]
\end{equation}


\subsection{Source term}
\hspace{5mm}Now we are trying to solve the heat equation with the source term:
\begin{equation}
\left \{ \begin{array}{llc}
\pdv{u}{t} =\frac{\partial^{2}u}{\partial x^{2}} + g(x, t)& x \in (0,L),  t>0 \\ \\
u(0,t) = 0, u(L, t) = 0& t>0 \\ \\
u(x, 0) = f(x) & 0 \leq x \leq L
\end{array}\right.
\end{equation}\par
First we recall the homogeneous BVP
$$
\left \{ \begin{array}{llc}
\pdv{u}{t} =\frac{\partial^{2}u}{\partial x^{2}} & & x \in (0,L),  t>0 \\
u(0,t) = 0, u(L, t) = 0 && t>0 
\end{array}\right.
$$\par 
We know the eigenvalues and eigenfunctions:
$$
\left\{
 \begin{array}{lcc}
 \beta_{n} = - \left(
\frac{n \pi}{L}
\right)^{2} &, n \in N \\
V_{n}(x) = sin(\frac{n\pi}{L}x)
 \end{array} \right.
$$\par
We are looking for the solution of the form:
\begin{equation}
u(x, t) = \sum_{n=1}^{\infty}\tilde{T}_{n}(t)V_{n}(x)
\end{equation}\par
Plugging Eq(31) into the source term Eq(17), we get
$$
\sum_{n=1}^{\infty}\frac{d}{dt}(\tilde{T}_{n}(t))V_{n}(x) = \sum_{n=1}^{\infty}\tilde{T}_{n}(t) \frac{d^{2}}{dx^{2}}(V_{n}(x)) + g(x, t)
$$
with knowing that 
$$
V''(x) = \beta V(x)
$$\par
Now using the orthogonal property, we multiply both sides by $V_{m}$ and integrate from 0 to $L$:
$$
\sum_{n=1}^{\infty}\frac{d}{dt}(\tilde{T}_{n}(t))\int_{0}^{L}V_{n}(x)V_{m}(x)dx = \sum_{n=1}^{\infty}\tilde{T}_{n}(t)\beta_{n} \int_{0}^{L}V_{n}(x)V_{m}(x)dx  + \int_{0}^{L}g(x, t)V_{m}(x)dx 
$$
Using Eq(15)
$$
\int_{0}^{L}V_{n}(x)V_{m}(x)dx = 
\left\{
\begin{array}{lcc}
0 & if & m \neq n \\
\frac{L}{2} & if& m = n
\end{array}
\right.
$$\par
We have 
$$
\frac{d}{dt}(\tilde{T}_{m}(t)) = \beta_{m}\tilde{T}_{m}(t) + \frac{2}{L}\int_{0}^{L}g(x, t)V_{m}(x)dx
$$
we get an ODE
$$
\frac{d}{dt}(\tilde{T}_{n}(t)) = \beta_{n}\tilde{T}_{n}(t) +h_{n}(t)
$$\par
Then we consider the initial condition $u(x, 0) = f(x)$:
$$
u(x, 0) = \sum_{n = 1}^{\infty}\tilde{T}_{n}(0) V_{n}(x) = f(x)
$$\par
Like before, we multiply both sides by $V_{m}$ and integrate from 0 to $L$:
$$
\sum_{n=1}^{\infty}\tilde{T}_{n}(0) \int_{0}^{L}V_{n}(x)V_{m}(x)dx = \int_{0}^{L}f(x)V_{m}(x)dx
$$
$$
\tilde{T}_{n}(0) = \frac{2}{L}\int_{0}^{L}f(x)V_{n}(x)dx = \omega_{n}
$$
Now we have the IVP for $\tilde{T}_{n}(t)$:
\begin{equation}
\left\{
\begin{array}{lcc}
\frac{d}{dt}\tilde{T}_{n}(t) = \beta_{n}\tilde{T}_{n}(t) + h_{n}(t) \\
\\
\tilde{T}_{n}(0) =  \omega_{n}
\end{array}\right.
\end{equation}where
$$
h_{n}(t) = \frac{2}{L}\int_{0}^{L}g(x, t)V_{n}(x)dx
$$
$$
\omega_{n} = \frac{2}{L}\int_{0}^{L}f(x)V_{n}(x)dx
$$\par
Now we need to solve the IVP problem with the variation of parameter formula:
$$
\tilde{T}_{n}(t) = \omega_{n}e^{\beta_{n}t} + \int_{0}^{t}e^{\beta_{n}(t-s)}h_{n}(s)ds
$$
Finally we find the complete solution of the heat equation with source term:
\begin{equation}
\begin{multlined}
u(x, t) = \sum_{n = 1}^{\infty} \frac{2}{L}\int_{0}^{L}f(x')sin(\frac{n\pi}{L}x')dx'e^{ - \left(
\frac{n \pi}{L}
\right)^{2}t}sin(\frac{n\pi}{L}x)\\
+
\sum_{n=1}^{\infty}\left[ 
\int_{0}^{t}e^
{- \left(
\frac{n \pi}{L}
\right)^{2}(t-s)
}\frac{2}{L}\int_{0}^{L}g(x, s)sin(\frac{n\pi}{L}x)ds
\right]
\end{multlined}
\end{equation}




\subsection{Non-homogeneous b.c.}
\hspace{5mm}Now we consider the fully nonhomogeneous problem, which means non-homogeneous source term in the equation (source term) and also in the boundary conditions.\par
Let's consider the following equation:
\begin{equation}
\left \{ \begin{array}{llc}
\pdv{u}{t} =\frac{\partial^{2}u}{\partial x^{2}} + g(x, t)& x \in (0,L),  t>0 \\ 
\\
u(0,t) = u_{1}(t), u(L, t) = u_{2}(t)& t>0 \\ 
\\
u(x, 0) = f(x) & 0 \leq x \leq L
\end{array}\right.
\end{equation}here the boundary condition is called the non-homogeneous Dirichlet b.c.\par
Let's consider a new function $\theta(x, t)$:
$$
\theta(x, t) = u(x, t) -w(x, t)
$$Let $w(0, t) = u_{1}(t)$ and $w(L, t) = u_{2}(t)$, we can have the new $\theta$ function to fit the homogeneous boundary conditions.\par
It is always a good choice to choose the linear relation. Thus, we can have:
$$
w(x, t) = \frac{L-x}{L}u_{1}(t) + \frac{x}{L}u_{2}(t)
$$\par
We can plug $\theta$ inside the function to get:
$$
\frac{\partial \theta}{\partial t} + \frac{L-x}{L}u'_{1}(t)+\frac{x}{L}u'_{2}(t) = \frac{\partial^{2} \theta}{\partial x^{2}} + g(x, t)
$$\par
Thus, the problem is transferred back to the source term problem:
\begin{equation}
\left \{ \begin{array}{llc}
\pdv{\theta}{t} =\frac{\partial^{2}\theta}{\partial x^{2}} + \tilde{g}(x, t)& x \in (0,L),  t>0 \\ 
\\
\theta(0,t) = 0, \theta(L, t) = 0& t>0 \\ 
\\
\theta(x, 0) = \tilde{f}(x) & 0 \leq x \leq L
\end{array}\right.
\end{equation}
$$
\tilde{g}(x, t) = g(x, t) -\frac{L-x}{L}u'_{1}(t)-\frac{x}{L}u'_{2}(t)
$$
$$
\tilde{f}(x, t) = f(x)-\frac{L-x}{L}u_{1}(0)-\frac{x}{L}u_{2}(0)
$$




\subsection{Wave equation}
\hspace{5mm}
Now we consider the homogeneous wave equation:
\begin{equation}
\left \{ \begin{array}{llc}
\frac{\partial^{2} u}{\partial t^{2}} =c^{2}\frac{\partial^{2}u}{\partial x^{2}} & x \in (0,L),  t>0 \\ 
\\
u(0,t) = 0, u(L, t) = 0& t>0 \\ 
\\
u(x, 0) = \phi_{1}(x), \pdv{u}{t} =\phi_{2}(x) & 0 \leq x \leq L
\end{array}\right.
\end{equation}\par
First, we consider the homogeneous BVP:
$$
\left \{ \begin{array}{llc}
\frac{\partial^{2} u}{\partial t^{2}} =c^{2}\frac{\partial^{2}u}{\partial x^{2}} & x \in (0,L),  t>0 \\ 
\\
u(0,t) = 0, u(L, t) = 0& t>0 \\ 
\end{array}\right.
$$
\par
The homogeneous BVP wave equation has the following general solution set:
$$
u_{n}(x, t) = V_{n}(x)T_{n}(x)=\sin(\frac{n\pi}{L}x)\left[a_{n}\cos(\frac{cn\pi}{L}t) +
 b_{n}\sin(\frac{cn\pi}{L}t)\right], n\in N
$$\par
Now, back to the IBVP Eq(23), in order to  satisfy the initial condition, we are looking for the solution in the form of:
\begin{equation}
u(x, t) = \sum_{n=1}^{\infty}\sin(\frac{n\pi}{L}x)\left[a_{n}\cos(\frac{cn\pi}{L}t) +
 b_{n}\sin(\frac{cn\pi}{L}t)\right]
\end{equation}
\par
We know that when $t = 0$, we have $\sin(\frac{cn\pi}{L}t)=0$ and $\cos(\frac{cn\pi}{L}t)=1$. Thus we have:
\begin{equation}
u(x, 0)=
\sum_{n=1}^{\infty}\sin(\frac{n\pi}{L}x)\cdot a_{n} = \phi_{1}(x)
\end{equation}
\begin{equation}
u'(x, 0) =
\sum_{n=1}^{\infty}\sin(\frac{n\pi}{L}x)\left[\frac{cn\pi}{L}
 b_{n}\right] = \phi_{2}(x)
\end{equation}\par
Recall the orthogonality we have:
$$
\int_{0}^{L}\sin(\frac{n\pi}{L}x)\sin(\frac{m\pi}{L}x) = \left\{
\begin{array}{lll}
 0& m\neq n \\
\frac{L}{2} & m= n
\end{array}\right.
$$ \par
Thus, we can find that:
$$
a_{n}=\frac{2}{L}\int_{0}^{L}\phi_{1}(x)\sin(\frac{n\pi}{L}x)dx
$$
$$
b_{n}=\frac{2}{cn\pi}\int_{0}^{L}\phi_{2}(x)\sin(\frac{n\pi}{L}x)dx
$$\par
To summarize, the solution to the IBVP is$u(x, t)$:
$$
\sum_{n=1}^{\infty}\sin(\frac{n\pi}{L}x)\left[\frac{2}{L}\int_{0}^{L}\phi_{1}(x')\sin(\frac{n\pi}{L}x')dx'\cos(\frac{cn\pi}{L}t) +
\frac{2}{cn\pi}\int_{0}^{L}\phi_{2}(x')\sin(\frac{n\pi}{L}x')dx'\sin(\frac{cn\pi}{L}t)\right]
$$\par
In the case with non-homogeneous B.C., we'll do the change of variable to make the boundary condition homogeneous as in the heat equation case. In the case with non-homogeneous source term, we need to find a particular solution for the inhomogeneous BVP using variation of parameter formula. \par
Boundary conditions determine the shape of the eigenfunctions. \\
\textbf{Dirichlet} boundary conditions (with rectangular domain in 1D) we have the \textbf{Fourier sine series}:
$$
\left\{ \sin(\frac{n\pi}{L}x) | n\in N
\right\}
$$\\
\textbf{Neuman} boundary conditions, we have \textbf{Fourier cosine series}:
$$
\left\{ \cos(\frac{n\pi}{L}x) | n\in N
\right\}
$$\\
For periodic boundary condtions, we will get both the sine and cosine eigenfunctions.\par
For non-rectangular domain, the eigenfunctions may not be trigonometric functions. \textbf{Can One Hear the Shape of a Drum?}\par


\subsection{Problems in higher(spatial) dimentions}
\hspace{5mm}
Now we consider a 2D(spatial) heat equation:
\begin{equation}
\left\{
\begin{array}{lll}
u_{t} = u_{xx} + u_{yy} & 0<x<1, 0<y<1, t>0\\
 \\
 u(0, y, t)=0,  u_{x}(1, y, t)=0 \\
  u(x, 0, t)=0,  u(x, 1, t)=0 & t >0 \\ \\
 u(x, y, 0) = f(x, y)
\end{array}\right.
\end{equation}note that we have a Neumann boundary condition.
\par
we also do the separation of variable:
$$
u(x, y, t) = X(x)Y(y)T(t)
$$plugging it into the 2D heat equation, we get:
$$
X(x)Y(y)T'(t) = X''(x)Y(y)T(t) + X(x)Y''(y)T(t)
$$
$$
\frac{T'(t)}{T(t)}= \frac{X''(x)}{X(x)}+\frac{Y''(y)}{Y(y)} = \beta
$$\par
Thus, we have:
\begin{equation}
\frac{T'(t)}{T(t)}= \beta
\end{equation}
$$
\frac{X''(x)}{X(x)}= \beta - \frac{Y''(y)}{Y(y)} = \mu
$$
\begin{equation}
\left\{
\begin{array}{lll}
X''(x) = \mu X(x)\\
Y''(y) = (\beta - \mu)Y(y)
\end{array}\right.
\end{equation}\par
Consider the boundary conditions:
$$
u(0, y, t)=0 \Rightarrow X(0)=0
$$
$$u_{x}(1, y, t) \Rightarrow X'(1) = 0
$$
$$
u(x, 0, t)=0 \Rightarrow Y(0) = 0
$$
$$
u(x, 1, t)=0 \Rightarrow Y(1) = 0
$$\par
We end up with two eigenvalue problems.
$$
\left\{
\begin{array}{lll}
X''(x) = \mu X(x)\\
X(0)=0, X'(1)=0
\end{array}\right.
$$
$$
\left\{
\begin{array}{lll}
Y''(y) = (\beta - \mu)Y(y)\\
Y(0)=0, Y(1)=0
\end{array}\right.
$$\par
Then we have the eigenvalue and eigenfunction as such:
$$
\mu_{n} = -(n\pi -\frac{\pi}{2})^{2}
$$
$$
X_{n}(x) = \sin((n\pi - \frac{\pi}{2})x), n \in N
$$then for each $\mu = \mu_{n}$, we have:
$$
\left\{
\begin{array}{lll}
Y''(y) = (\beta -\mu_{n})Y(y)\\
Y(0)=Y(1)=0
\end{array}\right.
$$\par
The eigenvalue and eigenfunctions are:
$$
\beta - \mu_{n} = -(m\pi)^{2}
$$
$$
Y_{m}(y) = \sin(m\pi y), m \in N
$$
note that $\beta$ depends on two indices $m$ \& $n$:
$$
\beta_{m, n} =  -(n\pi -\frac{\pi}{2})^{2} -(m\pi)^{2}
$$\par
Now back to Eq(28), we have the time term:
$$
T'_{m,n}(t) =  -[(n\pi -\frac{\pi}{2})^{2} +(m\pi)^{2}]T_{m,n}(t)
$$
\begin{equation}
T_{m,n}(t) = C_{m,n}e^{\beta_{m,n}t}=C_{m,n}e^{\ -[(n\pi -\frac{\pi}{2})^{2} +(m\pi)^{2}]t}
\end{equation}\par
Now for the BVP we have:
$$
u(x, y, t) = \sum_{m,n = 1}^{\infty}C_{m,n}e^{\ -[(n\pi -\frac{\pi}{2})^{2} +(m\pi)^{2}]t} \sin((n\pi - \frac{\pi}{2})x)\sin(m\pi y)
$$\par
At time $t=0$, we have the initial condition
$$u(x, y, 0) = f(x, y) = \sum_{m,n = 1}^{\infty}C_{m,n}\sin((n\pi - \frac{\pi}{2})x)\sin(m\pi y)
$$\par
To determine $C_{m,n}$ we it by $X_{p}(x)Y_{q}(y)$ for the orthogonality and integrate it:
$$
\int_{0}^{1}\int_{0}^{1}f(x, y)X_{p}(x)Y_{q}(y)dxdy =  \sum_{m,n = 1}^{\infty}C_{m,n}X_{n}(x)X_{p}(x)Y_{m}(y)Y_{q}(y)dxdy = \frac{1}{4}C_{p, q}
$$
Thus we have:
$$
C_{p, q} = 4\int_{0}^{1}\int_{0}^{1}f(x, y)X_{p}(x)Y_{q}(y)dxdy
$$




\subsection{Laplace equation}
\hspace{5mm}
Now let's consider the following equation:
\begin{equation}
\left\{
\begin{array}{lll}
u_{xx} + u_{yy} = 0 & 0<x<1, 0<y<1, t>0\\
 \\
 u(x, 0)=0,  u(x, 1)=x - x^{2} & 0 \leq x \leq 1\\ \\
  u(0,y)=0,  u(1, y)=0 &  0 \leq y \leq 1
\end{array}\right.
\end{equation}\par
Note that we no longer have the initial conditions, since there is no ''time'' dependence. The solutions to the Laplace equation $\Delta u = 0$ are the steady-state solutions to the heat equation. \par
Assuming that $u(x, y) = X(x)Y(y)$, we'll have:
$$
X''(x)Y(y)+X(x)Y''(y) =0
$$
$$
 \frac{X''(x)}{X(x)}+\frac{Y''(y)}{Y(y)} =0
$$
$$
 \frac{X''(x)}{X(x)}=-\frac{Y''(y)}{Y(y)}=\beta
$$\par
$$
\left\{
\begin{array}{lll}
X''(x) = \beta X(x)\\
Y''(y) = -\beta Y(y)
\end{array}\right.
$$
Note that we have three homogeneous boundary conditions:
$$
\begin{array}{lll}
u(x, 0) = 0 \Rightarrow Y(0) = 0\\
u(0, y) = 0 \Rightarrow X(0) = 0\\
u(1, y) = 0 \Rightarrow X(1) = 0\\
u(x, 1) = x-x^{2} 
\end{array}
$$\par
From above, we get:
$$
\left\{
\begin{array}{lll}
X''(x) = \beta X(x)\\
X(0)=0, X(1)=0
\end{array}\right.
$$ with the eigenvalue and eigenfunction
$$
\beta_{n} = -(n\pi)^{2}
$$
$$
X_{n}(x) = \sin(n\pi x), n \in N
$$\par
For the $y$ variable:
$$
Y''(y) = (n\pi)^{2}Y(y)
$$The eigenfunction is:
$$
Y_{n}(y)=C_{1}e^{n\pi y} +C_{2}e^{-n\pi y}
$$Using the third boundary condition $Y(0) = 0$, we have $C_{1}+C_{2}$:
$$
Y_{n}(y)= C_{1}(e^{n\pi y}-e^{-n\pi y})=2C_{1}\sinh(n\pi y)
$$\par
Now with three homogeneous boundary conditions, we have the solution:
$$
u_{n}(x, y) = \sum_{n = 1}^{\infty} C_{n}\sin(n\pi x)\sinh(n\pi y), n \in N
$$\par
Now we need  to choose $C_{n}$ to statisfy the remaining boundary condition $u(x, 1) = x-x^{2}$.
$$
 \sum_{n = 1}^{\infty} C_{n}\sin(n\pi x)\sinh(n\pi y) = x-x^{2}
$$multiply both side by $\sin(n\pi x)$ and integrate.\par
$$
\frac{1}{2}C_{n}\sinh(n\pi) = \int_{0}^{1}(x-x^{2})\sin(n\pi x)dx
$$
$$
C_{n}=\frac{ \int_{0}^{1}(x-x^{2})\sin(n\pi x)dx}{2\sinh(n\pi) }
$$
 Finally, we have:
 $$
 u(x, y)=\sum_{n=1}^{\infty}\frac{ \int_{0}^{1}(x-x^{2})\sin(n\pi x)dx}{2\sinh(n\pi) }\sin(n\pi x)\sinh(n\pi y) 
 $$\par
 Note that in the case we have more than one inhomogeneous boundary condition, we need to perform the \textbf{change of variables} to convert the equation to a problem with three homogeneous boundary conditions.\par



\subsection{Inner product space}
\hspace{5mm}
Note that the definition of a vector space gives an algebraic structure for vectors (i.e., addition and scalar multiplication). In the following, we shall build a geometric structure for vectors, such as length, distance, angle, etc.\par
First, we introduce the notation of \textit{inner product}, which is a generalization of  the dot product on the Euclidean space R$^{n}$.\\
\textbf{Definition 5.1.} Let $V$ be a real vector space. Consider a function from $V \times V$ to R that maps any pair of $u, v \in V$ to a real number. The function is called an \textbf{inner product} donated by $\langle \cdot, \cdot \rangle $(i.e.,$\langle u, v \rangle $ is a real number) if for any $u, v, w \in V$ and $\alpha \in$ R it satisfies:
$$
\langle u, v \rangle  = \langle v, u \rangle 
$$ 
$$
\langle u+w, v \rangle = \langle u, v \rangle + \langle w, v \rangle 
$$
$$
\langle \alpha u, v \rangle = \alpha \langle u, v \rangle 
$$
$$
\langle u, v \rangle \ge0,   \hspace{3mm}and \hspace{3mm}   \langle u, v \rangle  = 0 \hspace{3mm} if  \hspace{3mm}u=0
$$\par
Let us define next the notation of \textit{lengths}, which is called \textbf{norm} on the given vector space. \\
\textbf{Definition 5.2.}Consider a function form a vector space $V$ to R. The function is called a \textbf{norm} denoted by $\| \cdot \|$ if it satisfies:
$$
\| u\| \ge0\hspace{3mm}and \hspace{3mm} \| u\|=0\hspace{3mm} if  \hspace{3mm}u=0
$$
$$
\|\alpha u\| = |\alpha|\| u\| , \hspace{3mm} for\hspace{2mm} all\hspace{2mm} \alpha \in R
$$
$$
\| u+v\| \le \| u\|+\| v\|
$$\par
A vector space on which a norm is defined is then called a normed space or normed vector space.\\
\textbf{Theorem 5.1.} Given an inner product on a vector space $V$, the function defined by $\|u\| = \sqrt{\langle u, u \rangle}$ for all $u \in V$ is a norm. \\
\textbf{Theorem 5.2.} (Cauchy-Schwarz inequality). For any $u,v \in V$, it holds that:
$$
|\langle u, v \rangle| \le \|u \| \|v \|=\sqrt{\langle u, u \rangle}\sqrt{\langle v, v \rangle}
$$\\
\textbf{Example 5.1.} in R$^{n}$ the dot product is an inner product:
$$
\langle u, v \rangle = u \cdot v = u_{1}v_{1}+u_{2}v_{2}+\cdot \cdot \cdot+u_{n}v_{n}
$$
The corresponding Euclidean norm is given by:
$$
\| u\| = \sqrt{\langle u, u \rangle}=\sqrt{u_{1}^{2}+u_{2}^{2}+\cdot \cdot \cdot+u_{n}^{2}}
$$\\
\textbf{Example 5.2.} Let $\hat{L}^{2} (\Omega; \mathbb{R})$ be the subspace of $C(\Omega; \mathbb{R})$ defined by:
$$
\hat{L}^{2} (\Omega; \mathbb{R}) = \{ u \in C (\Omega; \mathbb{R}) | \int_{\Omega}|u(x)|^2 dx < + \infty \}
$$Define that
$$
\langle f,g\rangle =\int_{\Omega}f(x)g(x)dx, \hspace{10mm} \forall  f,g \in \hat{L}^{2} (\Omega; \mathbb{R})
$$ is an inner product on $\hat{L}^{2} (\Omega; \mathbb{R})$ and the corresponding norm is given by:
$$
\|f\|=(\int_{\Omega}f(x)^{2}dx)^{1/2}
$$
\par
 $\hat{L}^2(\Omega; \mathbb{R})$ is a function space defined on a domain $\Omega\subset\mathbb{R}^n$ that is a subspace of the space of continuous functions $C(\Omega; \mathbb{R})$.\par
 Specifically, $\hat{L}^2(\Omega; \mathbb{R})$ is the space of square-integrable functions on $\Omega$, which means that a function $f:\Omega\rightarrow \mathbb{R}$ belongs to $\hat{L}^2(\Omega; \mathbb{R})$ if and only if its square is integrable on $\Omega$, i.e., if the integral of $|f(x)|^2$ over $\Omega$ is finite. More formally, we define $\hat{L}^2(\Omega; \mathbb{R})$ as the set of functions $f:\Omega\rightarrow \mathbb{R}$ that satisfy

$$\int_\Omega |f(x)|^2 dx < \infty,$$

where $dx$ denotes the integration measure on $\Omega$.

The functions in ${L}^2(\Omega; \mathbb{R})$ are generally not required to be continuous, although they may have some continuity properties depending on the domain $\Omega$.

The subspace $\hat{L}^2(\Omega; \mathbb{R})$ is a Hilbert space with respect to the inner product defined as

$$\langle f,g \rangle_{L^2(\Omega)} = \int_\Omega f(x)g(x)dx,$$

where $f,g\in \hat{L}^2(\Omega; \mathbb{R})$. This inner product induces a norm on $\hat{L}^2(\Omega; \mathbb{R})$ given by

$$\|f\|_{L^2(\Omega)} = \sqrt{\langle f,f \rangle_{L^2(\Omega)}}.$$\par
 
 
 
\textbf{What's the difference between $L^2(\Omega; \mathbb{R})$ and $\hat{L}^2(\Omega; \mathbb{R})$?}\par
$L^2(\Omega; \mathbb{R})$ and $\hat{L}^2(\Omega; \mathbb{R})$ are both function spaces that consist of square-integrable functions on a domain $\Omega\subseteq \mathbb{R}^n$. The main difference between them is in their function spaces.

$L^2(\Omega; \mathbb{R})$ is the space of all square-integrable functions on $\Omega$, regardless of their regularity properties. In particular, functions in $L^2(\Omega; \mathbb{R})$ need not be continuous or have any other regularity properties. The norm on $L^2(\Omega; \mathbb{R})$ is given by

$$\|f\|_{L^2(\Omega)} = \sqrt{\int_{\Omega} |f(x)|^2 dx},$$

which measures the size of a function in terms of its $L^2$ norm, or equivalently, the square root of the integral of its square over $\Omega$.

On the other hand, $\hat{L}^2(\Omega; \mathbb{R})$ is a subspace of $L^2(\Omega; \mathbb{R})$ consisting of square-integrable functions on $\Omega$ that are also continuous on $\Omega$. That is, $\hat{L}^2(\Omega; \mathbb{R})$ consists of functions that are both square-integrable and continuous on $\Omega$. The norm on $\hat{L}^2(\Omega; \mathbb{R})$ is also given by the $L^2$ norm, but with the restriction that we consider only continuous functions:

$$\|f\|_{\hat{L}^2(\Omega)} = \sqrt{\int_{\Omega} |f(x)|^2 dx}.$$

In summary, the main difference between $L^2(\Omega; \mathbb{R})$ and $\hat{L}^2(\Omega; \mathbb{R})$ is that the former contains all square-integrable functions on $\Omega$, while the latter contains only those square-integrable functions that are also continuous on $\Omega$.\\
\textbf{Orthogonal bases}. Let us first introduce the notion of \textit{angle}. Recall the following property of the dot product on $\mathbb{R}^{n}$:
$$
u \cdot v = \|u\| \|v\| \cos (\theta)
$$where $theta \in [0, \pi]$ is the angle formed by the two vectors $u$ and $v$. This can be generalized to any inner product space in a natural way by defining:
$$
\cos(\theta) = \frac{\langle u, v\rangle}{\|u\| \|v\|}
$$where $u$ and $v$ are two vectors in a given inner product space $V$. Note that it follows from the Cauchy-Schwarz inequality that the ratio $ \frac{\langle u, v\rangle}{\|u\| \|v\|}$ is indeed between -1 and 1.\\
\textbf{Definition 5.3} Let V be a vector space with an inner product $\langle \cdot,\cdot \rangle$. Two vector $u$ and $v$ in $V$ are called orthogonal (or perpendicular) if:
$$
\langle u, v\rangle =0
$$\par
A set of vectors $\{v_{1}, v_{2}, ...v_{n}$ in $V$ are orthogonal if:
$$
\langle v_{i}, v_{j}\rangle = 0 \hspace{3mm} for \hspace{3mm} any \hspace{3mm} i \neq j
$$\par
\textbf{Theorem 5.3} Let $V$ be a finite-dimensional inner product space with dimension $n$; and $\mathbb{B} = \{ e_{1}, e_{2}, ..., e_{n}\}$ be an orthonormal basis of $V$. Then for any $u \in V$, it holds that:
$$
u = \langle u, e_{1}\rangle e_{1}+\langle u, e_{2}\rangle e_{2}+ ...+\langle u, e_{n}\rangle e_{n} = \sum_{i=1}^{n}\langle u, e_{i}\rangle e_{i}
$$and 
$$
\|u\|^{2} = | \langle u, e_{1}\rangle  |^{2}+ | \langle u, e_{2}\rangle  |^{2}+...+| \langle u, e_{n}\rangle  |^{2}=\sum_{i=1}^{n}| \langle u, e_{n}\rangle  |^{2}
$$\par



\subsection{Fourier transform}
\hspace{5mm}
Motivation: One can think of the Fourier transform as an analogue of the Fourier series for functions defined on $\mathbb{R}$ instead of $[-L, L]$.\par
 Let's do a little calculation to see this is indeed the case. Remember for the heat equation on $[-L, L]$, if we have Dirichlet boundary condition we'll have the solution as Fourier sine series. If we have Neumann boundary condition, we'll have the solution as Fourier cosine series. \par
 For heat equation on $[-L, L]$ with period boundary condition (symmetry boundary condition), the eigenvalue problem becomes:
 $$
 \left\{
 \begin{array}{lll}
 V'(x) = \beta V(x) \\
 \\
 V(-L) = V(L)\\
 \\
V'(-L) = V'(L)
 \end{array}
 \right.
 $$\par
 The eigenvalue and eigenfunctions are:
 $$
 \beta_{n}= - \left( \frac{n\pi}{L}\right)^{2}, n=0, 1, 2
 $$
 $$
 V_{0} \equiv 1
 $$
 $$
  \left\{
 \begin{array}{lll}
 V_{n}^{c}(x) = \cos  \left( \frac{n\pi}{L}x\right)  \\
 \\
 V_{n}^{s}(x) = \sin  \left( \frac{n\pi}{L}x\right)
 \end{array}\right.
 $$\par
 These eigenfunctions form an orthogonal basis of 
 $$
 L^{2}([-L, L], \mathbb{R}) = \{ f, [-L, L] \to \mathbb{R} | \int_{-L}^{L} f^{2}(x) dx < \infty \}
 $$\par
 For each $f \in L^{2}([-L, L], \mathbb{R})$ it holds:
 \begin{equation}
 f(x) = \frac{\langle f, V_{0}\rangle}{\langle V_{0}, V_{0}\rangle}V_{0}+\sum_{n = 1}^{\infty}\frac{\langle f, V_{n}^{c}\rangle}{\langle V_{n}^{c}, V_{n}^{c}\rangle}V_{n}^{c}+\sum_{n = 1}^{\infty}\frac{\langle f, V_{n}^{s}\rangle}{\langle V_{n}^{s}, V_{n}^{s}\rangle}V_{n}^{s}
 \end{equation} this is based on the Theorem 5.3\\
 Also we have the orthogonormal part:
 $$
 \frac{\langle f, V_{n}^{c}\rangle}{\langle V_{n}^{c}, V_{n}^{c}\rangle}V_{n}^{c}=
 \langle f, \frac{V_{n}^{c}}{\sqrt{\langle V_{n}^{c}, V_{n}^{c}\rangle}}\rangle \frac{V_{n}^{c}}{\sqrt{\langle V_{n}^{c}, V_{n}^{c}\rangle}}
 $$
 where the inner product is:
 $$
\langle f, V_{n}\rangle = \int_{-L}^{L} f(x)V_{n}(x)dx
 $$\par
 Check that we have:
 $$
 \langle V_{n}^{i}, V_{n}^{i}\rangle =
 \left\{
 \begin{array}{lll}
 2L, & n =0\\
 L, & n\ge1, & i = c,s
 \end{array}
 \right.
 $$\par 
 So that we can rewrite Eq(32) as
 \begin{equation}
 f(x) = \frac{1}{2L}\int_{-L}^{L} f(x)V_{0}(x)dx+\frac{1}{L}
 \sum_{n=1}^{\infty} \int_{-L}^{L} f(x) \cos  \left( \frac{n\pi}{L}x\right) dx \cdot \cos  \left( \frac{n\pi}{L}x\right)+
 \end{equation} 
 $$
  \frac{1}{L}\sum_{n=1}^{\infty} \int_{-L}^{L} f(x) \sin  \left( \frac{n\pi}{L}x\right) dx \cdot \sin  \left( \frac{n\pi}{L}x\right)
 $$
 The RHS is the Fourier series.\par
 Fourier series, Fourier sine series, and Fourier cosine series are all methods for representing periodic functions in terms of trigonometric functions, but they differ in the types of functions they can represent.

A Fourier series represents a periodic function $f(x)$ with period $2\pi$ as a sum of trigonometric functions of the form $\cos(nx)$ and $\sin(nx)$, where $n$ is an integer. That is,

$$f(x) = \frac{a_0}{2} + \sum_{n=1}^\infty \left[a_n \cos(nx) + b_n \sin(nx)\right],$$

where the coefficients $a_0$, $a_n$, and $b_n$ are given by

$$a_0 = \frac{1}{\pi} \int_{-\pi}^{\pi} f(x) dx,$$
$$a_n = \frac{1}{\pi} \int_{-\pi}^{\pi} f(x) \cos(nx) dx,$$
$$b_n = \frac{1}{\pi} \int_{-\pi}^{\pi} f(x) \sin(nx) dx.$$

Note that the Fourier series can represent any periodic function with period $2\pi$, whether it is even, odd, or neither.

A Fourier cosine series, on the other hand, represents a periodic function $f(x)$ with period $2\pi$ as a sum of cosine functions of the form $\cos(nx)$ only, that is

$$f(x) = \frac{a_0}{2} + \sum_{n=1}^\infty a_n \cos(nx),$$

where the coefficients $a_0$ and $a_n$ are given by

$$a_0 = \frac{1}{\pi} \int_{-\pi}^{\pi} f(x) dx,$$
$$a_n = \frac{1}{\pi} \int_{-\pi}^{\pi} f(x) \cos(nx) dx.$$

Note that in the Fourier cosine series, there are no sine terms. Therefore, a Fourier cosine series can only represent even periodic functions, i.e., functions that satisfy $f(-x) = f(x)$.

Finally, a Fourier sine series represents a periodic function $f(x)$ with period $2\pi$ as a sum of sine functions of the form $\sin(nx)$ only, that is

$$f(x) = \sum_{n=1}^\infty b_n \sin(nx),$$

where the coefficients $b_n$ are given by

$$b_n = \frac{2}{\pi} \int_{0}^{\pi} f(x) \sin(nx) dx.$$

Note that in the Fourier sine series, there are no cosine terms. Therefore, a Fourier sine series can only represent odd periodic functions, i.e., functions that satisfy $f(-x) = -f(x)$.

In summary, the main difference between Fourier series, Fourier cosine series, and Fourier sine series is the type of functions they can represent: Fourier series can represent any periodic function, Fourier cosine series can only represent even periodic functions, and Fourier sine series can only represent odd periodic functions.\par
Using Euler's formula $e^{ix} = \cos(x) + i\sin(x)$, we can rewrite the Fourier series as:
\begin{equation}
f(x) = \frac{1}{2L} \sum_{n= -\infty}^{\infty} \int_{-L}^{L} f(s)e^{-i(\frac{n\pi (s-x)}{L})}ds
\end{equation}
$$
e^{-i(\frac{n\pi (s-x)}{L})}=e^{-i(\frac{n\pi s}{L})}\cdot e^{i(\frac{n\pi x}{L})}=
 [\cos(\frac{n\pi s}{L}) - i\sin(\frac{n\pi s}{L})]\cdot [\cos(\frac{n\pi x}{L}) + i\sin(\frac{n\pi x}{L})]
$$
$$
=\cos(\frac{n\pi s}{L})\cos(\frac{n\pi x}{L})+i \cos(\frac{n\pi s}{L}) \sin(\frac{n\pi x}{L}) - i\cos(\frac{n\pi x}{L})\sin(\frac{n\pi s}{L})+\sin(\frac{n\pi s}{L})\sin(\frac{n\pi x}{L})
$$
$$
\int_{-L}^{L} f(s)e^{-i(\frac{n\pi (s-x)}{L})}ds=\int_{-L}^{L} f(s) [\cos(\frac{n\pi s}{L}) - i\sin(\frac{n\pi s}{L})]\cdot [\cos(\frac{n\pi x}{L}) + i\sin(\frac{n\pi x}{L})] ds
$$\par
Using Euler's formula $e^{ix} = \cos(x) + i\sin(x)$, we can rewrite the cosine and sine terms as complex exponentials:

$$\cos\left(\frac{n\pi}{L}s\right) = \frac{1}{2}\left(e^{i\frac{n\pi}{L}s} + e^{-i\frac{n\pi}{L}s}\right),\quad \sin\left(\frac{n\pi}{L}s\right) = \frac{1}{2i}\left(e^{i\frac{n\pi}{L}s} - e^{-i\frac{n\pi}{L}s}\right)$$

Substituting these into the expression for $f(x)$, we have:

\begin{align*}
f(x) &= \frac{1}{2L}\int_{-L}^{L} f(x)V_{0}dx + \frac{1}{L}\sum_{n=1}^{\infty} \int_{-L}^{L} f(s) \left(\frac{1}{2}\left(e^{i\frac{n\pi}{L}s} + e^{-i\frac{n\pi}{L}s}\right)\right) ds \cdot \left(\frac{1}{2}\left(e^{i\frac{n\pi}{L}x} + e^{-i\frac{n\pi}{L}x}\right)\right) \\
&+\frac{1}{L} \sum_{n=1}^{\infty} \int_{-L}^{L} f(s) \left(\frac{1}{2i}\left(e^{i\frac{n\pi}{L}s} - e^{-i\frac{n\pi}{L}s}\right)\right) ds \cdot \left(\frac{1}{2i}\left(e^{i\frac{n\pi}{L}x} - e^{-i\frac{n\pi}{L}x}\right)\right)
\end{align*} we can find that
$$
\frac{1}{4L}\sum_{n=1}^{\infty} \int_{-L}^{L} f(s) \left(\left(e^{i\frac{n\pi}{L}s} + e^{-i\frac{n\pi}{L}s}\right)\right) ds \cdot \left(\left(e^{i\frac{n\pi}{L}x} + e^{-i\frac{n\pi}{L}x}\right)\right)+
$$
$$
-\frac{1}{4L} \sum_{n=1}^{\infty} \int_{-L}^{L} f(s) \left(\left(e^{i\frac{n\pi}{L}s} - e^{-i\frac{n\pi}{L}s}\right)\right) ds \cdot \left(\left(e^{i\frac{n\pi}{L}x} - e^{-i\frac{n\pi}{L}x}\right)\right)
$$
$$
=
\frac{1}{2L}\sum_{n=1}^{\infty} \int_{-L}^{L} f(s) \left(\left(e^{i\frac{n\pi}{L}s}\right)\right) ds \cdot \left(\left( e^{-i\frac{n\pi}{L}x}\right)\right)+\frac{1}{2L}\sum_{n=1}^{\infty} \int_{-L}^{L} f(s) \left(\left(e^{-i\frac{n\pi}{L}s}\right)\right) ds \cdot \left(\left( e^{i\frac{n\pi}{L}x}\right)\right)
$$
$$
=
\frac{1}{2L}\sum_{n=1}^{\infty} \int_{-L}^{L} f(s) \left(\left(e^{i\frac{n\pi}{L}s}\right)\right) ds \cdot \left(\left( e^{-i\frac{n\pi}{L}x}\right)\right)+\frac{1}{2L}\sum_{-\infty}^{n =-1} \int_{-L}^{L} f(s) \left(\left(e^{i\frac{n\pi}{L}s}\right)\right) ds \cdot \left(\left( e^{-i\frac{n\pi}{L}x}\right)\right)
$$


note that when $n = 0$
$$
\frac{1}{2L}\int_{-L}^{L} f(x)V_{0}dx = \left(\frac{1}{2L}\int_{-L}^{L}f(s) e^{-i\frac{n\pi}{L}(s-x)} ds\right) e^{i\frac{n\pi}{L}x}=\frac{1}{2L}\int_{-L}^{L}f(s) V_{0} ds
$$
Simplifying this expression, we get:

$$f(x) = \frac{1}{2L} \sum_{-\infty}^{\infty} \int_{-L}^{L} f(s)e^{-i(\frac{n\pi (s-x)}{L})}ds$$\par
What we get if  $L \to \infty$?\\ 
To see what happens, we define:
$$
\omega_{n} = \frac{n\pi}{L}, \hspace{5mm} n \in \mathbb{Z}
$$
$$
\omega_{n+1}-\omega_{n}=\frac{\pi}{L}
$$as $L \to \infty$ the difference $\frac{\pi}{L} \to 0$ and $\omega_{n}$ becomes continuous.\par
Thus, we can write the $\sum$ as $\int$:
\begin{equation}
f(x) = \frac{1}{2L} \sum_{-\infty}^{\infty} \int_{-L}^{L} f(s)e^{-i(\frac{n\pi (s-x)}{L})}ds
\end{equation}
\begin{align*}
&= \frac{1}{2\pi} \left( \frac{\pi}{L} \right) \sum_{-\infty}^{\infty} \int_{-L}^{L} f(s)e^{-i(\frac{n\pi (s-x)}{L})}ds\\
&=  \frac{1}{2\pi} \int_{-\infty}^{\infty}\int_{-\infty}^{\infty} f(s)e^{-i\omega (s-x)}dsd\omega
\end{align*}\par
To summarize, we have:
\begin{equation}
f(x) = \frac{1}{\sqrt{2\pi}}\int_{-\infty}^{\infty} F(\omega) e^{i\omega x} d\omega
\end{equation}with
\begin{equation}
F(\omega) = \frac{1}{\sqrt{2\pi}}\int_{-\infty}^{\infty} f(s) e^{-i\omega s}ds
\end{equation} which are the \textbf{Fourier transform} and \textbf{inverse Fourier transform}. 























\end{document}