% This is a template for doing homework assignments in LaTeX

\documentclass{article} % This command is used to set the type of document you are working on such as an article, book, or presenation

\usepackage{geometry} % This package allows the editing of the page layout
\usepackage{amsmath}  % This package allows the use of a large range of mathematical formula, commands, and symbols
\usepackage{graphicx}  % This package allows the importing of images
\usepackage{mathtools}
\usepackage[T1]{fontenc}
\usepackage{lmodern}
\newcommand{\question}[2][]{\begin{flushleft}
        \textbf{Question #1}: {#2}

\end{flushleft}}
\newcommand{\sol}{\textbf{Solution}:} %Use if you want a boldface solution line
\newcommand{\maketitletwo}[2][]{\begin{center}
        \Large{\textbf{Assignment #1}       
        \vspace{5pt}
            } % Name of course here
        \vspace{5pt}
        
        \normalsize{Wenge Huang  % Your name here
        
        \today}        % Change to due date if preferred
        \vspace{15pt}
        
\end{center}}
\begin{document}
    \maketitletwo[1]  % Optional argument is assignment number
    %Keep a blank space between maketitletwo and \question[1]
    
    \question[1]{Find the Fourier sine expansion of the following function
    \begin{equation}
    f(x) = x^{2},  x\in [0, 1]
    \end{equation}
    } 
    $Solution$: We assume $$f(x) = \sum_{n=1}^{\infty} B_{n} \sin (n\pi x) = x^{2}$$\par
    The coefficient of $\sin(n\pi x)$ in Fourier sine series is:
    \begin{align*}
B_{n} &= 2\int_{0}^{1} x^{2} \sin(n \pi x)dx \\
                 &= \frac{2(-1)^{n-1}}{n\pi}+\frac{4}{(n\pi)^{3}}((-1)^{n}-1)
    \end{align*}\par
    Therefore, the Fourier sine expansion of the given function is
    $$
    x^{2} = \sum_{n=1}^{\infty}\left[ \frac{2(-1)^{n-1}}{n\pi}+\frac{4}{(n\pi)^{3}}((-1)^{n}-1)\right]\sin(n\pi x), x\in [0,1]
    $$
    
    
    
    
    \question[2]{Consider the following IBVP (initial boundary value problem) for the 1D heat equation posed on the initerval [0, L] for some L > 0:}
\begin{equation}
\left \{ \begin{array}{llc}
\frac{\partial u(x, t)}{\partial t} =\frac{\partial^{2}u (x,t)}{\partial x^{2}} & x \in (0,L),  t>0 \\
\\
u_{x}(0,t) = 0, u_{x}(L, t) = 0& t>0 \\
\\
u(x, 0) = x & 0 \leq x \leq L
\end{array}\right.
\end{equation}\par
$Solution$: We assume the solution is of form:
$$
u(x, t) = V(x)T(t)
$$
Then the equation is 
$$
T'(t)/T(t) = V''(x)/V(x) = \beta
$$
where $\beta$ is a constant. We solve $V$ by eignefunctions.\\
(a) If $\beta > 0$, 
$$
V(x) = c_{1}e^{-\sqrt{\beta}x}+c_{2}e^{\sqrt{\beta}x}
$$
$$
V'(x) = -\sqrt{\beta}c_{1}e^{-\sqrt{\beta}x}+\sqrt{\beta}c_{2}e^{\sqrt{\beta}x}
$$\par
By $V'(0) = V'(L) =0$, we have $c_{1} = c_{2} =0$. There is no non-trivial solutions.\\
(b)If $\beta =0$
$$
V(x) =c_{1} +c_{2}x 
$$
$$
V'(x) =c_{2}
$$
\par
By $V'(0) = V'(L) =0$, we have $c_{2} = 0$. The eigenfunction is $V_{0}(x) = 1$.\\
(c)
If $\beta <0$
$$
V(x) = c_{1}\cos(\sqrt{-\beta}x)+c_{2}\sin(\sqrt{-\beta}x)
$$    
$$
V'(x)=-c_{1}\sqrt{-\beta}\sin(\sqrt{-\beta}x)+c_{2}\sqrt{-\beta}\cos(\sqrt{-\beta}x)
$$\par
By $V'(0) = V'(L) =0$, we have $c_{2} = 0$, $\beta_{n} = -\left(\frac{n\pi}{L}\right)^{2}$, $n \in N$. The eigenfunction $V_{n} = \cos(\frac{n\pi}{L}x)$, $n \in N$. The corresponding $T_{n}(t) = e^{\beta_{n}t}= e^{ -\left(\frac{n\pi}{L}\right)^{2}t}$.\par
The solution can be written as:
$$
u(x, t) = c_{0} + \sum_{n=1}^{\infty}c_{n}V_{n}(x)T_{n}(t)=c_{0} + \sum_{n=1}^{\infty}c_{n}e^{ -\left(\frac{n\pi}{L}\right)^{2}t}\cos(\frac{n\pi}{L}x)
$$\par
The boundary value
$$
x =u(x, 0) = c_{0} + \sum_{n=1}^{\infty}c_{n}\cos(\frac{n\pi}{L}x)
$$
is exactly the Fourier cosine expansion of $x$ on $[0, L]$, thus
$$
c_{0}= \frac{1}{L}\int_{0}^{L}xdx=\frac{L}{2}
$$
$$
c_{n} = \frac{2}{L}\int_{0}^{L}x\cos(\frac{n\pi}{L}x)dx, n\in N
$$
Therefore, for $t>0$, $x \in [0, L]$,
$$
u(x, t) = \frac{L}{2}+\sum_{n=1}^{\infty}\left(\frac{2}{L}\int_{0}^{L}x\cos(\frac{n\pi}{L}x)dx \right)e^{ -\left(\frac{n\pi}{L}\right)^{2}t}\cos(\frac{n\pi}{L}x)
$$

\end{document}













